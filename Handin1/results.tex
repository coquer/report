\section{Results}

In order to test our results, we have built a testing framework. The requirements for this were, that it should allow us to quickly and accurately evaluate changes to the detection algorithms. Manual testing is good, but requires us to manually select a sequence and a frame, which can get tedious very quickly.

To combat this, and also get more consistent results we have defined an array of representative frames from each sequence. We have carefully selected the frames which we had found to cause most problems for our detection algorithms during manual testing. Therefore our representative sample contains the worst-case frames for each sequence (eye looking different directions, challenging light, or eye position). 

On the other hand, we have selected only 5-8 frames per sequence (56 frames together), so the testing can quickly finish and produce representative results. We have also made sure that all of the frames in question do contain the pupil (i.e no frames with closed eye or where no detection can be made) The testing script also writes all the frames with pupil/iris/glint detections drawn on top into a folder for further inspection.

Unfortunately we cannot automatically test if a detection is correct, because we have no reference implementation, so the correctness still has to be evaluated manually. But what we can detect automatically is the number of detected results (i.e what percentage of images that contained eye have had at least something detected by our implementation). This can give us  at least preliminary results, if we assume that most of the detected pupils are correct, we should get a good idea about the results from this alone. Of course this does not mean that we can solely rely on this automatic testing and manual verification still has to be performed.

\begin{tabular}{ |l|l|l|l| }
\hline
\multicolumn{4}{ |c| }{Evaluation} \\
\hline
Sequence & Frames Evaluated & Pupil Detected & Detected Correctly \\ \hline

eye1.avi & 5 & 5 & 5 \\ \hline
eye2.avi & 6 & 6 & 6 \\ \hline
eye3.avi & 8 & 7 & 6 \\ \hline
eye4.avi & 8 & 8 & 7 \\ \hline
eye5.avi & 6 & 3 & 2 \\ \hline
eye6.avi & 7 & 7 & 5 \\ \hline
eye7.avi & 8 & 3 & 3 \\ \hline
eye8.avi & 8 & 0 & 0 \\ \hline \hline

Total & 56 & 39 (69.64\%) & 34 (60.71\%) \\ \hline

\end{tabular}